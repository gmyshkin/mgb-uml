\chapter{Prometheus Grafana}
\small{\textit{-- Matthew Smith, Bowen Jiang, Gleb Myshkin}}
\index{prometheus grafana} 
\index{Chapter!Prometheus Grafana }
\label{Chapter Prometheus Grafana}

\begin{figure} [H]
\includegraphics[width=\textwidth]{png/Grafana.png}
  \centering
  \caption{Grafana Dashboard}
  \vspace{-0.3cm}
\end{figure}

\begin{figure} [H]
\includegraphics[width=\textwidth]{png/Prometheus.png}
  \centering
  \caption{Prometheus UI}
  \vspace{-0.3cm}
\end{figure}

The node-exporter runs on the machine and collected the system level metrics such as CPU usage and disk space. This data is then taken and exported in a format that Prometheus would understand. Prometheus takes the data from node-exporter at regular intervals then stores the data in its time-series database. Grafana connects to Prometheus as a data source, takes the data from it and shows it in easy to read graphs and charts on a dashboard. Dashboard 1860 shows a lot of data about the computer, such as how much RAM it's using or  how much Disk Space is being utilized. There is a lot more data that it can show but it can be pretty difficult to decipher all of it and understand what it's showing.