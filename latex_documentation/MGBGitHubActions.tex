\chapter{Github Actions}
\small{\textit{-- Matthew Smith, Bowen Jiang, Gleb Myshkin}}
\index{github actions} 
\index{Chapter!Github Actions }
\label{Chapter Github Actions}


To be able to connect our Overleaf with GitHub, we needed to modify our itManual.tex file to make sure that GitHub handles some situations that Overleaf does by itself. We added some of the following lines of code:

\begin{lstlisting}[style=linuxstyle, language=bash]
\usepackage[utf8]{inputenc}

\usepackage[T1]{fontenc}  % enables \textquotedbl and better hyphenation
\usepackage{textcomp}     % extra text symbols (keeps \textquotedbl robust)
% Map full-width comma (U+FF0C) to a normal comma for pdfLaTeX
\DeclareUnicodeCharacter{FF0C}{,}
\end{lstlisting}

The above lines of code were added to handle the correct Unicode character handling to make sure that it compiles on different systems.

\begin{lstlisting}[style=linuxstyle, language=bash]
\lstset{
  literate=
    {-}{{-}}1
    {--}{{--}}1
    {...}{{...}}1
    {-}{{-}}1
    {|}{{|}}1
    {<<}{{<<}}1  {>>}{{>>}}1
}
\end{lstlisting}

The above lines were also added to the main lstset command to map common Unicode symbols such as smart quotes or dashes to their LaTeX equivalents, preventing compiler errors.
The styles for lstdefinestyle were also moved from the end of the Overleaf file into the preamble in the GitHub overleaf file for better coding practices.

\begin{lstlisting}[style=linuxstyle, language=bash]
% ------- Front matter lists (SWAPPED COMMANDS) -------
\cleardoublepage
\pagenumbering{roman}
\setcounter{tocdepth}{2}   % sections in ToC

% Contents
\cleardoublepage
\phantomsection
\pdfbookmark[0]{Contents}{toc}
\contentspage

% List of Figures
\cleardoublepage
\phantomsection
\pdfbookmark[0]{List of Figures}{lof}
\figurelistpage

% List of Tables
\cleardoublepage
\phantomsection
\pdfbookmark[0]{List of Tables}{lot}
\tablelistpage
% -----------------------------------

\cleardoublepage
\end{lstlisting}

The above lines were added so that the Overleaf in GitHub would correctly generate the Table of Contents, List of Figures, and List of Tables with proper PDF bookmarks.

\begin{lstlisting}[style=linuxstyle, language=bash]
\setlength{\headheight}{27pt} % small bump to silence fancyhdr warnings
\addtolength{\topmargin}{-11.6pt}
\end{lstlisting}

The above lines were added after pagestyle{fancy} to fix some common layout warnings from the fancyhdr package.

\begin{lstlisting}[style=linuxstyle, language=bash]
#Before Glossary:
Code snippet

% ---- Back matter ----
\cleardoublepage
\phantomsection
\pdfbookmark[0]{Glossary}{glossary}

#Before Bibliography:
Code snippet

\cleardoublepage
\phantomsection
\pdfbookmark[0]{Bibliography}{bibliography}

#Before Index:
Code snippet

\cleardoublepage
\phantomsection
\pdfbookmark[0]{Index}{index}
\end{lstlisting}

The above lines were added respective to Glossary, Bibliography, and Index to ensure that they are also properly bookmarked in the PDF.


A new file was added to the GitHub repo (latex-ci.yml) so that it would automatically compile the LaTeX document through a Continuous Integration pipeline. Every time we would push code or create a pull request, this action would automatically run on a GitHub server, which would then compile the Overleaf document into a PDF and save it into a PDF to download. The following is the code that is inside the yml file:

\begin{lstlisting}[style=linuxstyle, language=bash]
name: LaTeX build (fast)

on:
  push:
    branches: [ main, master ]
  pull_request:
  workflow_dispatch: {}

concurrency:
  group: ${{ github.workflow }}-${{ github.ref }}
  cancel-in-progress: true

jobs:
  build:
    runs-on: ubuntu-latest
    timeout-minutes: 30
    container:
      image: ghcr.io/xu-cheng/texlive-full:latest

    steps:
      - name: Checkout
        uses: actions/checkout@v4

      # Create version.tex so \VERSION is not "Version DEV"
      - name: Write version.tex
        run: |
          SHORT_SHA=$(echo "${GITHUB_SHA}" | cut -c1-7)
          DATE=$(date -u +"%Y-%m-%d")
          echo "\newcommand{\VERSION}{Version ${DATE} (${SHORT_SHA})}" > version.tex
          echo "Wrote version.tex:"
          cat version.tex

      # Minted requires pygments
      - name: Install pygments
        run: |
          apk update
          apk add --no-cache python3 py3-pip py3-pygments

      # (Optional) start fresh to avoid stale .toc/.aux confusion
      - name: Clean LaTeX aux files
        run: |
          latexmk -C || true
          rm -rf _*.minted* || true

      # Compile; latexmk will rerun as needed for ToC/refs
      - name: Compile with latexmk
        id: compile
        continue-on-error: true
        run: |
          set -x
          latexmk -pdf -interaction=nonstopmode -file-line-error --shell-escape itManual.tex || true

      # Helpful diagnostics: show first LaTeX error (binary logs safe)
      - name: Show first LaTeX error
        if: always()
        run: |
          echo "---- Searching for first LaTeX error in itManual.log ----"
          if [ -f itManual.log ]; then
            LINE=$(grep -n -a -E '^(!|.*LaTeX Error:)' itManual.log | head -1 | cut -d: -f1 || true)
            if [ -n "$LINE" ]; then
              START=$((LINE>20 ? LINE-20 : 1))
              END=$((LINE+40))
              sed -n "${START},${END}p" itManual.log
            else
              echo "No obvious error lines; showing last 200 lines:"
              tail -n 200 itManual.log
            fi
          else
            echo "itManual.log not found."
          fi

      # Upload outputs no matter what so you can inspect locally
      - name: Upload PDF
        if: always()
        uses: actions/upload-artifact@v4
        with:
          name: itManual-pdf
          path: itManual.pdf
          if-no-files-found: warn
          retention-days: 14

      - name: Upload logs
        if: always()
        uses: actions/upload-artifact@v4
        with:
          name: latex-logs
          path: |
            itManual.log
            itManual.fls
            itManual.fdb_latexmk
            *.aux
            *.toc
            *.lof
            *.lot
            _*.minted*
          if-no-files-found: warn
          retention-days: 14
\end{lstlisting}


\begin{figure} [H]
\includegraphics[width=\textwidth]{png/GitHub Actions.png}
  \centering
  \caption{GitHub Actions Screen}
  \label{fig:github-actions}
  \vspace{-0.3cm}
\end{figure}

Figure \ref{fig:github-actions} is a screenshot of the GitHub actions screen that was setup through the yml file and other modifications of the itManual.tex file.